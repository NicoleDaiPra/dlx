\chapter{Simulation, Synthesis and Implementation}
\label{chap:syn_impl}

In this chapter we will discuss the flow we have followed from the DLX's simulation to implementation.

\section{Simulation}

The simulation has been performed on two tools: {\it Vivado} and {\it ModelSim}. The various components of the DLX have been developed
by using the former tool, which is also the one we used to perform unit testing. The latter has been used to perform the final integration
testing. In the {\it dlx\_sim} folder a simulation script for ModelSim can be found, called \verb|sim_script.tcl|.

This script first lets ModelSim to analyze the various VHDL files, then analyzed one of the provided testbenches and it runs a simulation
for 100 ns.

We did not provided the testbenches we have written for testing the components in isolation because, during the final tests, we had to change
many small things and we did not have enough time to modify them as well.

\subsection{Test files}

In the folder {\it tb} all the testbenches that can be used with \verb|sim_script.tcl| are stored. By running all the scripts we have provided
it is possible to verify the correctness of the entire DLX's ISA.

The files are:
\begin{enumerate}
    \item \verb|tb_arith.vhd|: it aims at testing all the various arithmetic instructions, either of type R and I, by also the forwarding logic.
    \item \verb|tb_branches.vhd|: it tests all the branches executed by the DLX along with the prediction abilities of the BTB.
    \item \verb|tb_jalr.vhd|: it verifies the correctness of the \verb|jalr| instruction. This test is separated because otherwise \verb|tb_branches| would have been too big.
    \item \verb|tb_jump_and_link.vhd|: it shows the ability of the processor to follows the ABI and to execute the \verb|jalr| instruction. As a bonus, it demonstrate the ability of the control unit and memory controller to handle cache misses.
    \item \verb|tb_load_store.vhd|: it tests if the processor is capable to handle all the supported loads and stores.
    \item \verb|tb_mult.vhd|: it tests if the multiplication is properly executed in various scenarios, like when it would need forwarding and if it stalls the pipeline correctly.
    \item \verb|tb_r.vhd|: it checks whether the processor is able to execute all the supported R instructions, with and without forwarding.
\end{enumerate}

The related assembly files can be found in the folder {\it test\_assembly}.

\section{Synthesis}

Each time we developed a new datapath component, we have synthetized it to assess if it was able to meet our initial frequency target of $1\ GHz$. Among all the datapath components,
the multiplier was the only one to not meet such standard, as it is able to reach a little bit more than $850\ MHz$ if synthetized alone.
When we synthetized the first version of the complete DLX (that is, also with the control unit and the memory controller) we used the \verb|set_dont_touch| command to

\begin{enumerate}
    \item use \verb|compile_ultra| on critical parts of the design, that is the EXE stage and the control unit;
    \item use \verb|compile| on the rest of the processor, as it less time-sensitive.
\end{enumerate}

The \verb|set_dont_touch| command, in fact, tells to the tool to not modify anymore an elaborated and compiled entity. This removes some optimization opportunities, but it is very handy
when dealing with big circuits such as a processor to reduce the synthesis time.
The result of the synthesis didn't meet our performance target, since it reached only $800 MHz$, but we still were satisfied. However, the more we have tested the DLX, the more we have found
issues that needed to be fixed. Most of the issue were related to the control unit and the stall management, therefore it would have needed a complete redesign. Unfortunately, there was not enough
time to do it, hence we have been forced to add some hacks to make it work. This has severely affected the DLX's overall performance, as $t_{CK}$ had to be raised to $1.9\ ns$. This results in
$f_{CK} \approx 525\ MHz$, which is almost half of our initial frequency target. This value has been obtained by synthetizing the whole DLX with the \verb|compile_ultra| command.

Below, the final time, area and power report are shown.

\subsection{Time report}

\begin{verbatim}
Startpoint: ctrl_u/curr_es_reg[1]
            (rising edge-triggered flip-flop clocked by clk)
  Endpoint: dp/id_exe_regs/b_mult_reg/curr_data_reg[23]
            (rising edge-triggered flip-flop clocked by clk)
  Path Group: clk
  Path Type: max

  Des/Clust/Port     Wire Load Model       Library
  ------------------------------------------------
  dlx_syn            5K_hvratio_1_1        NangateOpenCellLibrary

  Point                                                   Incr       Path
  --------------------------------------------------------------------------
  clock clk (rise edge)                                   0.00       0.00
  clock network delay (ideal)                             0.00       0.00
  ctrl_u/curr_es_reg[1]/CK (DFF_X1)                       0.00       0.00 r
  ctrl_u/curr_es_reg[1]/Q (DFF_X1)                        0.09       0.09 r
  U4305/ZN (NAND3_X1)                                     0.04       0.13 f
  U4627/ZN (NAND2_X1)                                     0.04       0.17 r
  U4054/ZN (AND4_X2)                                      0.07       0.23 r
  U4596/ZN (NAND4_X1)                                     0.05       0.29 f
  U3980/ZN (NOR2_X2)                                      0.07       0.36 r
  U3938/Z (BUF_X2)                                        0.05       0.41 r
  U4032/ZN (INV_X1)                                       0.04       0.45 f
  U7477/ZN (OAI22_X1)                                     0.05       0.50 r
  U7153/ZN (INV_X1)                                       0.02       0.52 f
  U7151/ZN (NAND2_X1)                                     0.03       0.55 r
  U4560/ZN (XNOR2_X1)                                     0.07       0.62 r
  U4341/ZN (NOR2_X1)                                      0.03       0.65 f
  U4558/ZN (NOR2_X1)                                      0.07       0.72 r
  U4550/ZN (NAND2_X1)                                     0.04       0.76 f
  U4549/ZN (NOR2_X1)                                      0.05       0.80 r
  U4529/ZN (AOI21_X1)                                     0.03       0.84 f
  U4466/ZN (INV_X1)                                       0.06       0.90 r
  U4459/ZN (AOI21_X1)                                     0.05       0.95 f
  U4323/ZN (OAI21_X1)                                     0.05       1.00 r
  U4322/ZN (XNOR2_X1)                                     0.06       1.06 r
  U4319/ZN (NAND2_X1)                                     0.03       1.09 f
  U4317/ZN (NOR2_X1)                                      0.04       1.13 r
  U3971/ZN (AND4_X2)                                      0.07       1.20 r
  U4452/ZN (NAND2_X1)                                     0.03       1.23 f
  U4450/ZN (AND2_X1)                                      0.04       1.27 f
  U4316/ZN (OAI21_X1)                                     0.04       1.32 r
  U4315/ZN (NAND2_X1)                                     0.04       1.35 f
  U4314/ZN (NAND2_X1)                                     0.04       1.40 r
  U3973/ZN (OR2_X2)                                       0.05       1.44 r
  U4443/ZN (AND2_X1)                                      0.04       1.48 r
  U4390/ZN (NAND2_X1)                                     0.04       1.52 f
  U4004/ZN (NAND2_X1)                                     0.06       1.59 r
  U6576/ZN (NAND2_X1)                                     0.07       1.66 f
  U4137/ZN (NOR2_X1)                                      0.11       1.77 r
  U3277/ZN (AOI22_X1)                                     0.05       1.82 f
  U3278/ZN (NAND3_X1)                                     0.04       1.86 r
  dp/id_exe_regs/b_mult_reg/curr_data_reg[23]/D (DFF_X1)
                                                          0.01       1.87 r
  data arrival time                                                  1.87

  clock clk (rise edge)                                   1.90       1.90
  clock network delay (ideal)                             0.00       1.90
  dp/id_exe_regs/b_mult_reg/curr_data_reg[23]/CK (DFF_X1)
                                                          0.00       1.90 r
  library setup time                                     -0.03       1.87
  data required time                                                 1.87
  --------------------------------------------------------------------------
  data required time                                                 1.87
  data arrival time                                                 -1.87
  --------------------------------------------------------------------------
  slack (MET)                                                        0.00
\end{verbatim}

As it can be seen, the critical path starts in the EXE part of the control unit, specifically from the register that holds the EXE FSM's state,
and terminates in the multiplication's register \verb|b| data.

\subsection{Area report}

\begin{verbatim}
Number of ports:                          600
Number of nets:                          6868
Number of cells:                         5918
Number of combinational cells:           5083
Number of sequential cells:               832
Number of macros:                           0
Number of buf/inv:                        899
Number of references:                      54

Combinational area:       10821.944148
Noncombinational area:    8594.991691
Net Interconnect area:      undefined  (Wire load has zero net area)

Total cell area:          19416.935839
Total area:                 undefined
\end{verbatim}

\subsection{Power report}

\begin{verbatim}
Global Operating Voltage = 1.1  
Power-specific unit information :
    Voltage Units = 1V
    Capacitance Units = 1.000000ff
    Time Units = 1ns
    Dynamic Power Units = 1uW    (derived from V,C,T units)
    Leakage Power Units = 1nW


  Cell Internal Power  =   6.9865 mW   (80%)
  Net Switching Power  =   1.7389 mW   (20%)
                         ---------
Total Dynamic Power    =   8.7254 mW  (100%)

Cell Leakage Power     = 463.1583 uW


                 Internal         Switching           Leakage            Total
Power Group      Power            Power               Power              Power   (   %    )  Attrs
--------------------------------------------------------------------------------------------------
io_pad             0.0000            0.0000            0.0000            0.0000  (   0.00%)
memory             0.0000            0.0000            0.0000            0.0000  (   0.00%)
black_box          0.0000            0.0000            0.0000            0.0000  (   0.00%)
clock_network     11.1759        1.1986e+03          114.8263        1.2098e+03  (  13.17%)
register       6.7667e+03           72.8469        1.4489e+05        6.9844e+03  (  76.01%)
sequential         0.0000            0.0000            0.0000            0.0000  (   0.00%)
combinational    208.6468          467.4612        3.1815e+05          994.2623  (  10.82%)
--------------------------------------------------------------------------------------------------
Total          6.9865e+03 uW     1.7389e+03 uW     4.6316e+05 nW     9.1886e+03 uW
\end{verbatim}

\subsection{Synthesis files}

The synthesis file are stored in the {\it dlx\_syn} folder. Inside the folder it is provided a script called
\verb|syn_script.tcl|, capable of running the the whole synthesis process.

\section{Implementation}