\chapter{Conclusions}
\label{chap:conclusions}

All in all, we have mixed feelings for this project. For a part we are satisfied by the amount of things we have been able to achieve,
since we managed to add a large instruction set, a BTB, two caches, the forwarding unit and the ABI support.

However, we regret the way we implemented the multiplier. Actually, we regret to have added it at all.

In the earliest phase
of development we have decided to rearrange the Booth's algorithm in such a way that it could be pipelined and would not take a lot of area.
Unfortunately, we did not have enough time to implement it properly and this forced us to do compromise in order to keep it, compromises we
would not have accepted by going back. It has hugely increased the complexity of the control unit and the amount of signals needed to handle
all the possible stalls and corner cases. It also took a lot of time from us for testing and it had many problems, time that we could have
dedicated to improving the rest of the system. This is all without even considering the fact that it is the only component in the EXE stage that,
synthesized by itself with \verb|compile_ultra|, is not able to reach the target frequency of $1\ GHz$.

The final frequency of our DLX is, indeed, "only" of $525\ MHz$, a decent value but far from our target. Sadly due to all the delays we had
we did not have time to remove it and test everything again so in the end we kept it.