\chapter{Introduction}
\label{chap:intro}

\section{General architecture}

The proposed DLX is a 5-stage, MIPS-based, scalar processor. Its implementation respects the MIPS ABI and O32 calling convention,
it supports all the basic instructions, most of the ones proposed as PRO and a few 32-bits ones coming from the MIPS64
architecture \cite{MIPS64_arch}. The DLX also features a 32 entries {\it BTB}, a 64 entries {\it four-way associative data cache} and a
{\it forwarding unit}. To support the ABI and the extended instruction set we had to modify the file \verb|dlxasm.pl|.
By going a little bit more in details, here there is a summary of what each stage does:

\begin{enumerate}
    \item \textbf{IF stage}: here is where the {\it program counter}, also referred to as PC, is stored. This stage contains the logic
    to update the value of the next PC and to choose the right PC at each clock cycle.
    
    \item \textbf{ID stage}: in this stage the instruction fetched with the PC calculated the cycle before is decoded and the register file is
    accessed for reading. In case an instruction has an immediate field it is extracted and properly extended.

    \item \textbf{EXE stage}: based on the decoding happened the cycle before, one of the ALU's functional units is activated to perform
    the requested calculation. It is worth noting that the ALU is capable of executing an instruction like \verb|jalr| in a single clock cycle
    instead of the 2 specified in the MIPS architecture manual \cite{MIPS64_arch}. Another exception is the \verb|mult| instruction which works
    with the {\it integer register file} (also referred to as RF) and it is a multi-cycle operation.

    \item \textbf{MEM stage}: if the operation in this stage is a load or a store the cache is accessed, and in case of a load shorter than
    32 bits sign extension is performed (if required). On the other hand, if an instruction does not fall in the two categories outlined before,
    its result is simply propagated to the next stage.

    \item \textbf{WB stage}: in this stage all the operations that produce a result to be stored in the RF can do so. The result to be written is
    chosen among the one coming from the cache and the one coming from the ALU.
\end{enumerate}

The datapath is managed by the {\it control unit}, which has been implemented as a mix of two approaches: the hardwired and the FSM. The reasoning behind
this choice is explained in chapter \ref{chap:cu}. Within the control unit there is also a stall unit, whose job is to detect any possible hazard, and to
either enable data forwarding or to force a bubble in the pipeline for stalling the processor.

Outside the control unit and the datapath there is the {\it memory controller}, which handles the communication between the RAM and the cache as well as cache misses.

\section{Supported ISA}

In table \ref{tab:sup_instr} are shown all the supported instructions:

\begin{table}[!ht]
    \centering
    \begin{tabular}{ |c|c|c|c||c|c|c|c| }
        \hline
        Name & Opcode & Func & Type & Name & Opcode & Func & Type \\

        \hline
        \verb|nop| & \verb|0x00| & \verb|0x00| & \verb|R| & \verb|sll| & \verb|0x00| & \verb|0x00| & \verb|R| \\
        \hline
        \verb|srl| & \verb|0x00| & \verb|0x06| & \verb|R| & \verb|sra| & \verb|0x00| & \verb|0x07| & \verb|R| \\
        \hline
        \verb|jr| & \verb|0x00| & \verb|0x08| & \verb|R| & \verb|jalr| & \verb|0x00| & \verb|0x09| & \verb|R| \\
        \hline
        \verb|mult| & \verb|0x00| & \verb|0x0E| & \verb|R| & \verb|mfhi| & \verb|0x00| & \verb|0x10| & \verb|R| \\
        \hline
        \verb|mflo| & \verb|0x00| & \verb|0x12| & \verb|R| & \verb|add| & \verb|0x00| & \verb|0x20| & \verb|R| \\
        \hline
        \verb|addu| & \verb|0x00| & \verb|0x21| & \verb|R| & \verb|sub| & \verb|0x00| & \verb|0x22| & \verb|R| \\
        \hline
        \verb|subu| & \verb|0x00| & \verb|0x23| & \verb|R| & \verb|and| & \verb|0x00| & \verb|0x24| & \verb|R| \\
        \hline
        \verb|or| & \verb|0x00| & \verb|0x25| & \verb|R| & \verb|xor| & \verb|0x00| & \verb|0x26| & \verb|R| \\
        \hline
        \verb|seq| & \verb|0x00| & \verb|0x28| & \verb|R| & \verb|sne| & \verb|0x00| & \verb|0x29| & \verb|R| \\
        \hline
        \verb|slt| & \verb|0x00| & \verb|0x2A| & \verb|R| & \verb|sgt| & \verb|0x00| & \verb|0x2B| & \verb|R| \\
        \hline
        \verb|sle| & \verb|0x00| & \verb|0x2C| & \verb|R| & \verb|sge| & \verb|0x00| & \verb|0x2D| & \verb|R| \\
        \hline
        \verb|sltu| & \verb|0x00| & \verb|0x3A| & \verb|R| & \verb|sgtu| & \verb|0x00| & \verb|0x3B| & \verb|R| \\
        \hline
        \verb|sleu| & \verb|0x00| & \verb|0x3C| & \verb|R| & \verb|sgeu| & \verb|0x00| & \verb|0x3D| & \verb|R| \\
        \hline
        \verb|bgez| & \verb|0x01| & \verb|0x00| & \verb|I| & \verb|bltz| & \verb|0x01| & \verb|0x00| & \verb|I| \\
        \hline
        \verb|j| & \verb|0x02| & \verb|0x00| & \verb|J| & \verb|jal| & \verb|0x03| & \verb|0x00| & \verb|J| \\
        \hline
        \verb|beq| & \verb|0x04| & \verb|0x00| & \verb|I| & \verb|bne| & \verb|0x05| & \verb|0x00| & \verb|I| \\
        \hline
        \verb|blez| & \verb|0x06| & \verb|0x00| & \verb|I| & \verb|bgtz| & \verb|0x07| & \verb|0x00| & \verb|I| \\
        \hline
        \verb|addi| & \verb|0x08| & \verb|0x00| & \verb|I| & \verb|addui| & \verb|0x09| & \verb|0x00| & \verb|I| \\
        \hline
        \verb|subi| & \verb|0x0A| & \verb|0x00| & \verb|I| & \verb|subui| & \verb|0x0B| & \verb|0x00| & \verb|I| \\
        \hline
        \verb|andi| & \verb|0x0C| & \verb|0x00| & \verb|I| & \verb|ori| & \verb|0x0D| & \verb|0x00| & \verb|I| \\
        \hline
        \verb|xori| & \verb|0x0E| & \verb|0x00| & \verb|I| & \verb|beqz| & \verb|0x10| & \verb|0x00| & \verb|I| \\
        \hline
        \verb|bnez| & \verb|0x11| & \verb|0x00| & \verb|I| & \verb|slli| & \verb|0x14| & \verb|0x00| & \verb|I| \\
        \hline
        \verb|srli| & \verb|0x16| & \verb|0x00| & \verb|I| & \verb|srai| & \verb|0x17| & \verb|0x00| & \verb|I| \\
        \hline
        \verb|seqi| & \verb|0x18| & \verb|0x00| & \verb|I| & \verb|snei| & \verb|0x19| & \verb|0x00| & \verb|I| \\
        \hline
        \verb|slti| & \verb|0x1A| & \verb|0x00| & \verb|I| & \verb|sgti| & \verb|0x1B| & \verb|0x00| & \verb|I| \\
        \hline
        \verb|slei| & \verb|0x1C| & \verb|0x00| & \verb|I| & \verb|sgei| & \verb|0x1D| & \verb|0x00| & \verb|I| \\
        \hline
        \verb|lb| & \verb|0x20| & \verb|0x00| & \verb|I| & \verb|lh| & \verb|0x21| & \verb|0x00| & \verb|I| \\
        \hline
        \verb|lw| & \verb|0x23| & \verb|0x00| & \verb|I| & \verb|lbu| & \verb|0x24| & \verb|0x00| & \verb|I| \\
        \hline
        \verb|lhu| & \verb|0x25| & \verb|0x00| & \verb|I| & \verb|sb| & \verb|0x28| & \verb|0x00| & \verb|I| \\
        \hline
        \verb|sh| & \verb|0x29| & \verb|0x00| & \verb|I|  & \verb|sw| & \verb|0x2B| & \verb|0x00| & \verb|I| \\
        \hline
        \verb|sltui| & \verb|0x3A| & \verb|0x00| & \verb|I| & \verb|sgtui| & \verb|0x3B| & \verb|0x00| & \verb|I| \\
        \hline
        \verb|sleui| & \verb|0x3C| & \verb|0x00| & \verb|I| & \verb|sgeui| & \verb|0x3D| & \verb|0x00| & \verb|I| \\
        \hline
    \end{tabular}
    \caption{Supported instructions}
    \label{tab:sup_instr}
\end{table}

It is worth pointing out that \verb|nop| and \verb|sll| have exactly the same opcode and func fields,
because the \verb|nop| actually corresponds to \verb|sll $zero, $zero, $zero| (\cite{MIPS64_arch}), which is an instruction that achieves nothing.
Also \verb|bltz| and \verb|bgez| share the same opcode (as stated in \cite{MIPS64_arch} for release 1 up to 5), as their difference lies in the
instruction's bits 20-16: they are set to \verb|0b00000| for the former one and to \verb|0b00001| for the latter one.
Finally, \verb|jr| and \verb|jalr| are considered R instructions as specified in the manual \cite{MIPS64_arch}.

\section{Design objectives}

Our goal during the design phase was to deliver a fast processor and to cut latencies of the most used general purpose instructions. Our frequency target
was to reach $1\ GHz$, and to be able to execute \verb|jal| and \verb|jalr| in a single clock cycle instead of the 2 specified in
\cite{MIPS64_arch}. We also felt that was important to provide to a potential user a complete instruction set, therefore we have introduced also \verb|mult|,
which is normally executed by the FPU. As discussed in chapter \ref{chap:exe}, achieving high clock frequencies with a multiplication unit is not trivial and
some compromises had to be done in term of latencies for this particular operation. We decided to trade the presence of the operation with latency, because as
stated in \cite{code_stats} multiplications accounts only for the $0.02\%$ of the utilized instructions.

\section{Folder organization}

In the root folder it is possible to find 3 folders, this file and the modified \verb|dlxasm.pl|. The folders are:

\begin{enumerate}
    \item \textbf{dlx\_sim}: all the files needed for the ModelSim simulation, as well as the files we used to test the DLX, can be found here.
    \item \textbf{dlx\_syn}: the synthetizable files are stored here along with the synthesis report.
    \item \textbf{dlx\_impl}: it contains all the files and the outputs of the physical design.
\end{enumerate}

The VHDL files in dlx\_syn are a subset of the ones present in dlx\_sim. Moreover, in dlx\_sim there are two additional folders:

\begin{enumerate}
    \item \textbf{tb}, that contains the VHDL testbenches needed to perform the actual simulation on ModelSim.
    \item \textbf{test\_assembly}, where the \verb|.asm| and \verb|.mem| files are saved. These files are read by their related testbench and are used to fed the {\it IROM}.
\end{enumerate}